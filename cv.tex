%%%%%%%%%%%%%%%%%%%%%%%%%%%%%%%%%%%%%%%%%
% Medium Length Professional CV
% LaTeX Template
% Version 2.0 (8/5/13)
%
% This template has been downloaded from:
% http://www.LaTeXTemplates.com
%
% Original author:
% Trey Hunner (http://www.treyhunner.com/)
%
% Important note:
% This template requires the resume.cls file to be in the same directory as the
% .tex file. The resume.cls file provides the resume style used for structuring the
% document.
%
%%%%%%%%%%%%%%%%%%%%%%%%%%%%%%%%%%%%%%%%%

%----------------------------------------------------------------------------------------
%   PACKAGES AND OTHER DOCUMENT CONFIGURATIONS
%----------------------------------------------------------------------------------------

\documentclass{resume} % Use the custom resume.cls style
\usepackage{hyperref}
\usepackage[left=0.75in,top=0.6in,right=0.75in,bottom=0.6in]{geometry} % Document margins

\name{Georgii Oblapenko} % Your name
\address{Ph.D. (Mechanics of Fluid, Gas and Plasma)}
\address{University of Texas at Austin \\ Oden Institute for Computational Engineering and Sciences}
\address{+79533767636 \\ kunstmord@kunstmord.com \\ kunstmord.com}

\begin{document}


%----------------------------------------------------------------------------------------
%   WORK EXPERIENCE SECTION
%----------------------------------------------------------------------------------------

\begin{rSection}{Experience}

\begin{rSubsection}{University of Texas at Austin}{January 2019 -- Present}{Postdoctoral Research Fellow}{Oden Institute for Computational Engineering and Sciences}
\item Development of a hybrid Discrete Velocity Method/Direct Simulation Monte Carlo code
\item Rarefied gas flow modelling
\end{rSubsection}


\begin{rSubsection}{University of Texas at Austin}{July 2018 -- January 2019}{Scientific Consultant}{Oden Institute for Computational Engineering and Sciences}
\item Development of a hybrid Discrete Velocity Method/Direct Simulation Monte Carlo code
\end{rSubsection}

\begin{rSubsection}{Saint-Petersburg State University}{September 2015 -- December 2017}{Assistant engineer}{Department of Hydroaeromechanics}
\item Lead developer of a C++ library aimed at kinetic theory computations
\item Implementation of state-to-state models in Direct Simulation Monte Carlo (DSMC) codes
\item Development of simplified models for vibrational relaxation rates
\item Numerical modeling of reaction rates in viscous gas flows
\end{rSubsection}

\begin{rSubsection}{Saint-Petersburg State University}{April 2013 -- September 2015}{Assistant researcher}{Department of Hydroaeromechanics}
\item Development of theoretical models of reaction rates in viscous gas flows
\item Numerical modeling of reaction rates in viscous gas flows
\end{rSubsection}

\begin{rSubsection}{GODA New Media Artist Collective}{October 2016 -- Present}{Programmer, sound designer}{}
\item Design of interactive installations, concept creation
\item Programming (Python, MAX/MSP, Arduino)
\item Sound design, soundtrack work
\end{rSubsection}

\begin{rSubsection}{Freelance}{September 2013 -- June 2014}{Web developer}{}
\item Creation of educational online games
\item Creation of online interface for an interactive remote laboratory
\end{rSubsection}

% {\bf Internships:}
\begin{rSubsection}{Internships}{}{}{}
\item \textbf{DLR, Göttingen, Germany (November 2017--January 2018)}.
Implementation of modern models of physico-chemical process rates in the DLR-TAU solver.

\item \textbf{The Federal University of Parana, Curitiba, Brazil (October 2016).}
Studied the influence of variable diameters of vibrationally excited molecules on relaxation processes.

\item \textbf{Khristianovich Institute of Theoretical and Applied Mechanics, Novosibirsk, Russia (February 2016).}
Worked on implementation of state-to-state models of non-equilibrium processes in DSMC code.
\end{rSubsection}

% \begin{enumerate}
% \end{enumerate}

%------------------------------------------------



\end{rSection}

%----------------------------------------------------------------------------------------
%   TECHNICAL STRENGTHS SECTION
%----------------------------------------------------------------------------------------
\pagebreak
\begin{rSection}{Technical skills}
{\bf Brief overview:}

\begin{tabular}{ @{} >{\bfseries}l @{\hspace{6ex}} l }
Computer languages & Python, C++, Fortran, Julia, MATLAB \\
& MAX/MSP, \LaTeX, Javascript, HTML, CSS \\
Tools & Unix command-line tools, Paraview \\ 
Skills & Numerical modeling, machine learning, digital signal processing \\
Online profiles & https://github.com/knstmrd \\
& https://github.com/godacollective \\
& https://www.kaggle.com/kunstmord \\
& https://www.linkedin.com/in/george-oblapenko
\end{tabular}

{\bf Detailed:}

\begin{rSubsection}{Numerical modeling}{}{}{}
\item Experience in implementing various numerical algorithms in C++, Python, Fortran, Julia and MATLAB 
\item Well-acquainted with numerical/scientific libraries for Python (Numpy, Scipy) and C++ (Armadillo, GSL, Boost numerical libraries)
\end{rSubsection}

\begin{rSubsection}{Machine learning}{}{}{}
\item Experience in working with data-processing tools, various classification and regression algorithms in Python (Pandas, scikit-learn, XGBoost, LightGBM, Catboost); experience with audio processing, computer vision (OpenCV), and text processing (NLTK, Gensim)
\item Experience with deep learning (PyTorch, Keras)
\item Regular participant in Kaggle competitions (14 to date)
\end{rSubsection}

\begin{rSubsection}{Kinetic theory}{}{}{}
\item Experience in application of kinetic theory to non-equilibrium flow modeling (rate coefficient and transport properties computation)
\item One of the lead developers of the {\bf \href{https://github.com/lkampoli/kappa}{KAPPA}} kinetic theory library (C++)
\item Experience in CFD simulations of strongly non-equilibrium reacting gas flows and implementation of various models of thermo-chemical relaxation in CFD code
\item Experience in DSMC modeling of non-equilibrium rarefied gas flows and development of DSMC and Discrete Velocity Method codes
\end{rSubsection}

\begin{rSubsection}{Digital Signal Processing}{}{}{}
\item Experience in DSP in Python (Librosa, YAAFE)
\item Experience in implementation of various audio processing algorithms in C++, Python, MAX/MSP
\end{rSubsection}

\begin{rSubsection}{Miscellaneous}{}{}{}
\item Web development: familiar with Django and Django REST framework, experience working with Javascript, React, HTML, CSS
\item iOS app development: basic knowledge of Swift, AudioKit, SpriteKit
\end{rSubsection}
\pagebreak
\begin{rSection}{Education}

{\bf Saint-Petersburg State University} \hfill {\em April 2017} \\ 
Ph.D. degree (in Physical and Mathematical Sciences)\\
Area of research: Mechanics of Fluids, Gases and Plasma \\
Department of Hydroaeromechanics \smallskip \\
Research supervisor: Prof. Kustova E.V. \\

{\bf Saint-Petersburg State University} \hfill {\em June 2015} \\ 
Masters degree (with excellence) in Mechanics and Mathematical Modelling\\
Area of specialization: Molecular Kinetic Theory of Fluids and Gases\\
Department of Hydroaeromechanics \smallskip \\
Research supervisor: Prof. Kustova E.V. \\

{\bf Saint-Petersburg State University} \hfill {\em June 2013} \\ 
Bachelors degree in Mathematics and Mechanics \\
Department of Hydroaeromechanics \smallskip \\
Research supervisor: Prof. Kustova E.V. \\

\end{rSection}


\begin{rSection}{Research grants}

Participant of 3 Saint-Petersburg University grants, 1 Russian Science Foundation grant and 3 Russian Foundation for Basic Research grants (not including personal grants).

Participant of the ESA research project ``Exploring angular-momentum phenomenology in aerothermodynamics and MHD'' (as a subcontractor to the DLR), 2015--2016.

{\bf Own research and travel grants:}

\begin{enumerate}

    \item Research project ``Improvements of the thermo-chemical relaxation model used by the DLR-TAU code'' jointly sponsored by the DAAD and Saint-Petersburg State University, 2017--2018
    \item Research project ``Influence of variable diameters of vibrationally excited molecules on relaxation processes in strongly non-equilibrium gas flows'' jointly sponsored by the Santander Bank and Saint-Petersburg State University, 2016
    \item Research project ``Implementation of models of vibrational transitions in direct simulation methods'', 2015--2016, sponsored by the Russian Foundation for Basic Research
    \item Saint-Petersburg State University travel grants (2013, 2014, 2017)

\end{enumerate}

{\bf Stipends and awards:}
\begin{itemize}
    \item Stipend of the Russian President for students and PhD students studying disciplines corresponding to the prioritized areas of modernization of Russian economics (2017)
    \item Stipend of the Russian Government for students and PhD students studying disciplines corresponding to the prioritized areas of modernization of Russian economics (2016)
    \item Stipend of the Russian Government for students and PhD students (2016)
    \item Winner of the Saint-Petersburg Government Grant Competition for Students and Graduate Students (2013)
\end{itemize}
\end{rSection}


\pagebreak

\begin{rSection}{Publications, conference participation}
Author and co-author of 14 publications in SCOPUS/Web of Science-indexed peer-reviewed journals, participant of 7 international and 7 all-Russian conferences.

{\bf Publications in SCOPUS/Web of Science-indexed journals:}

\begin{enumerate}

\item \emph{Oblapenko G.P.} Calculation of Vibrational Relaxation Times Using a Kinetic Theory Approach // The Journal of Physical Chemistry A, 2018.

\item \emph{Campoli L., Oblapenko G.P., Kustova E.V.} KAPPA: Kinetic approach to physical processes in atmospheres library in C++ // Computer Physics Communications, 2018.

\item \emph{Istomin V.A., Oblapenko G.P.} Transport coefficients in high-temperature ionized air flows with electronic excitation // Physics of Plasmas, 2018.

\item \emph{Kremer G.M., Kunova O.V., Kustova E.V., Oblapenko G.P.} The influence of vibrational state-resolved transport coefficients on the wave propagation in diatomic gases // Physica A: Statistical Mechanics and its Applications, 2018.

\item \emph{Shoev, G., Oblapenko, G., Kunova, O., Mekhonoshina, M., \& Kustova, E.} Validation of vibration-dissociation coupling models in hypersonic non-equilibrium separated flows // Acta Astronautica, 2018.

\item \emph{Kustova E.V., Mekhonoshina M.A., Oblapenko G.P.} On the applicability of simplified state-to-state models of transport coefficients // Chemical Physics Letters, 2017.

\item \emph{Oblapenko G.P., Kashkovsky A.V., Bondar Ye.A.} State-to-state models of vibrational relaxation in Direct Simulation Monte Carlo (DSMC) // Journal of Physics: Conference Series, 2017.

\item  \emph{Kustova E.V., Oblapenko G.P.} Vibration-dissociation Coupling in Multi-Temperature Viscous Gas Flows // AIP Conference Proceedings, 2016. V. 1786. P. 150004 (1–8)

\item  \emph{Baikov B.S., Bayalina D.K., Kustova E.V., Oblapenko G.P.} Inverse Laplace Transform as a Tool for Calculation of State-specific Cross Sections of Inelastic Collisions // AIP Conference Proceedings, 2016.

\item  \emph{Shoev G.V., Bondar Ye.A., Oblapenko G.P., and Kustova E.V.} Development and testing of a numerical simulation method for thermally nonequilibrium dissociating flows in ANSYS Fluent // Thermophysics and Aeromechanics, 2016.

\item  \emph{Kustova E.V., Oblapenko G.P.} Mutual effect of vibrational relaxation and chemical reactions in viscous multitemperature flows // Physical Review E -- Statistical, Nonlinear, and Soft Matter Physics, 2016.

\item  \emph{Kustova E.V., Nagnibeda E.A., Oblapenko G.P., Savelev A.S., Sharafutdinov I.Z.} Advanced models for vibrational–chemical coupling in multi-temperature flows // Chem. Phys., 2016.

\item  \emph{Kustova E.V., Oblapenko G.P.} Reaction and internal energy relaxation rates in viscous thermochemically non-equilibrium gas flows // Phys. Fluids, 2015.

\item  \emph{Kustova E.V., Oblapenko G.P.} Rates of VT Transitions and Dissociation and Normal Mean Stress in a Non-equilibrium Viscous Multitemperature N$_2$/N Flow // AIP Conference Proceedings, 2014.


\end{enumerate}


{\bf Other publications:}

\begin{enumerate}
    \item  \emph{Kustova E.V., Oblapenko G.P., Sharafutdinov I.Z.} Vibrational relaxation models for non-equilibrium multi-temperature flows // Physico-chemical Kinetics in Gas Dynamics, 2015. Vol. 16. (In Russian)

    \item  \emph{Kustova E.V., Oblapenko G.P.} Vibrational relaxation rates in multi-temperature gas flows // Physico-chemical Kinetics in Gas Dynamics, 2014. Vol. 15. P. 1-4. (In Russian)

    \item  \emph{Kustova E.V., Oblapenko G.P.} Normal mean stress and rates of slow process in chemically and vibrationally non-equilibrium multi-temperature gas flows // Vestn. S.-Peterb. Univ., Ser 1, 2013. P. 111-120. (In Russian)
\end{enumerate}

{\bf Conference and school participation:}

\begin{enumerate}
    \item Tallinn University course on Experimental Interaction Design (Physiological Computing Technologies for Performative Arts), 2018 (Saint-Petersburg, Russia)
    \item All-Russian conference on hydroaeromechanics, dedicated to S.V. Vallander's 100th anniversary, 2017 (Saint-Petersburg, Russia)
    \item International conference ``7th European Conference for Aeronautics and Space Sciences'', 2017 (Milan, Italy)
    \item International EUCASS workshop ``Aerospace Thematic Workshops: Fundamentals of Aerodynamic Flow and Combustion Control by Plasmas'', 2017 (Pushkin, Russia)
    \item All-Russian school-seminar ``Aerophysics and physical mechanics of classical and quantum systems'', 2016 (Moscow, Russia)
    \item International conference ``30th International Symposium on Rarefied Gas Dynamics'', 2016 (Victoria BC, Canada)
    \item All-Russian school-seminar ``Aerophysics and physical mechanics of classical and quantum systems'', 2015 (Moscow, Russia)
    \item All-Russian seminar ``XXIV All-Russian seminar with international partnership on jet, separation, and non-stationary flows'', 2015 (Novosibirsk, Russia)
    \item International conference ``International Scientific Conference On Mechanics ``The Seventh Polyakhonv's Reading'''', 2015 (Saint-Petersburg, Russia)
    \item All-Russian school-seminar ``Aerophysics and physical mechanics of classical and quantum systems'', 2014 (Moscow, Russia)
    \item International conference ``29th International Symposium on Rarefied Gas Dynamics'', 2014 (Xi'an, China)
    \item All-Russian conference ``Modern problems in rarefied gas dynamics'', 2013 (Novosibirsk, Russia)
    \item International conference ``9th IFAC Symposium on Advances in Control Education'', 2012 (Nizhniy Novgorod, Russia)
    \item All-Russian conference ``XIV Conference of Young Scientists on Navigation and Motion Control'', 2012 (Saint-Petersburg, Russia)
\end{enumerate}

\end{rSection}

\pagebreak

{\bf Kaggle competition participation and results}

\begin{tabular}{ @{} >{\bfseries}l @{\hspace{6ex}} l }
Instant Gratification, 2019 & Top 8\% \\
Freesound Audio Tagging, 2019 & Top 37\% \\
Quora Insincere Question Classification, 2019 & Top 22\% \\
Freesound General-Purpose Audio Tagging Challenge, 2018 & Top 11\% \\
Avito Demand Prediction Challenge, 2018 & Top 28\% \\
WSDM - KKBox's Music Recommendation Challenge, 2017 & Top 46\% \\
Sberbank Russian Housing Market, 2017 & Top 66\% \\
TalkingData Mobile User Demographics, 2016 & Top 60\% \\
Liberty Mutual Group: Property Inspection Prediction, 2015 & Top 55\% \\
Otto Group Product Classification Challenge, 2015 & Top 7\% \\ 
Driver Telematics Analysis, 2015 & Top 15\%(part of a team) \\ 
Galaxy Zoo - The Galaxy Challenge, 2014 & Top 48\% (part of a team) \\
The Marinexplore and Cornell University & \\
Whale Detection Challenge, 2013 & Top 23\%
\end{tabular}


\end{rSection}


%----------------------------------------------------------------------------------------
%   EDUCATION SECTION
%----------------------------------------------------------------------------------------


\end{document}
