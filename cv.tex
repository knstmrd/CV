
%----------------------------------------------------------------------------------------
%   PACKAGES AND OTHER DOCUMENT CONFIGURATIONS
%----------------------------------------------------------------------------------------

\documentclass{resume} % Use the custom resume.cls style

\usepackage[left=0.75in,top=0.6in,right=0.75in,bottom=0.6in]{geometry} % Document margins
\newcommand{\tab}[1]{\hspace{.2667\textwidth}\rlap{#1}}
\newcommand{\itab}[1]{\hspace{0em}\rlap{#1}}
\name{Georgii Oblapenko} % Your name
\address{Ph.D. (Physical and Mathematical Sciences) \\ Postdoctoral Researcher}
\address{German Aerospace Center (DLR), G\"{o}ttingen}
\address{georgii.oblapenko@dlr.de}
% \address{georgii.oblapenko@dlr.de}
% \address{}
% \address{University of Texas at Austin \\ Oden Institute for Computational Engineering and Sciences}

\begin{document}


%----------------------------------------------------------------------------------------
%   WORK EXPERIENCE SECTION
%----------------------------------------------------------------------------------------

\begin{rSection}{Work Experience}


\begin{rSubsection}{University of Texas at Austin}{September 2022 -- \textbf{Present}}{Scientific consultant}{Oden Institute for Computational Engineering and Sciences / Department of Department of Aerospace Engineering and Engineering Mechanics}
Mentoring of Master student developing PIC-DSMC methods for high-voltage gap closure modeling. Development of new PIC-DSMC algorithms.
\end{rSubsection}

\begin{rSubsection}{German Aerospace Center (DLR)}{November 2021 -- \textbf{Present}}{Postdoctoral Researcher}{Institute of Aerodynamics and Flow Technology, Spacecraft Department}
Uncertainty quantification of hypersonic expanding continuum flows. Development of new Direct Simulation Monte Carlo methods for plasma simulation. Development of a photon Monte Carlo radiation transport code. Development of moment method models for gases with internal degrees of freedom. Work on low-rank methods for rarefied gas dynamic applications. Supported by a personal grant of the Alexander von Humboldt Foundation.
\end{rSubsection}

\begin{rSubsection}{University of Texas at Austin}{July 2018 -- June 2021}{Postdoctoral Research Fellow}{Oden Institute for Computational Engineering and Sciences / Department of Department of Aerospace Engineering and Engineering Mechanics}
Development of a hybrid Discrete Velocity Method/Direct Simulation Monte Carlo code. Development of new Direct Simulation Monte Carlo algorithms for modeling of ionized rarefied gas flows. Development of new kinetic models for gases with internal degrees of freedom and chemically reacting gas flows.
Research project supported by Sandia National Laboratories.
% Development of Discrete Velocity Method and Direct Simulation Monte Carlo codes for modeling of ionized rarefied gas flows. Development of new kinetic models for gases with internal degrees of freedom and chemically reacting gas flows.
\end{rSubsection}


% \begin{rSubsection}{University of Texas at Austin}{July 2018 -- January 2019}{Scientific Consultant}{Oden Institute for Computational Engineering and Sciences  / Department of Department of Aerospace Engineering and Engineering Mechanics}
% Development of a hybrid Discrete Velocity Method/Direct Simulation Monte Carlo code.
% \end{rSubsection}

\begin{rSubsection}{Saint-Petersburg State University}{September 2015 -- December 2017}{Research engineer}{Department of Hydroaeromechanics}
Lead developer of a C++ library aimed for kinetic theory computations. Implementation of state-to-state models in Direct Simulation Monte Carlo (DSMC) codes, development of simplified models for vibrational relaxation rates, and numerical modeling of reaction rates in viscous gas flows.
\end{rSubsection}

\begin{rSubsection}{Saint-Petersburg State University}{April 2013 -- September 2015}{Assistant researcher}{Department of Hydroaeromechanics}
Development of theoretical models of reaction rates in viscous gas flows and their numerical modeling.
\end{rSubsection}

% \begin{rSubsection}{GODA New Media Artist Collective}{October 2016 -- Present}{Programmer, sound designer}{}
% \item Design of interactive installations, concept creation
% \item Programming (Python, MAX/MSP, Arduino)
% \item Sound design, soundtrack work
% \end{rSubsection}

% \begin{rSubsection}{Freelance}{September 2013 -- June 2014}{Web developer}{}
% \item Creation of educational online games
% \item Creation of online interface for an interactive remote laboratory
% \end{rSubsection}

% {\bf Internships:}
\begin{rSubsection}{Internships}{}{}{}
\item \textbf{DLR, G\"{o}ttingen, Germany (November 2017--January 2018)}.
Implementation of a new  vibrational relaxation model for air mixtures in the DLR-TAU CFD solver.

\item \textbf{Federal University of Parana, Curitiba, Brazil (October 2016).}
Study of wave propagation processes in high-temperature gas flows.

\item \textbf{Khristianovich Institute of Theoretical and Applied Mechanics, Novosibirsk, Russia (February 2016).}
Implementation of state-to-state models of non-equilibrium processes in a DSMC code.
\end{rSubsection}
\end{rSection}
\newpage
%----------------------------------------------------------------------------------------
%   EDUCATION SECTION
%----------------------------------------------------------------------------------------

\begin{rSection}{Education}

{\bf Saint-Petersburg State University} \hfill {\em September 2015 --- April 2017} \\ 
\textbf{Ph.D. degree} (in Physical and Mathematical Sciences)\\
Area of specialization: Mechanics of Fluids, Gases and Plasma \smallskip  \\
Department of Hydroaeromechanics \\
Research supervisor: Prof. E.V. Kustova
% Dissertation title: Physico-chemical relaxation rates in viscous non-equilibrium gas flows

{\bf Saint-Petersburg State University} \hfill {\em September 2013 --- June 2015} \\ 
\textbf{Masters degree (with excellence)} in Mechanics and Mathematical Modeling\\
Area of specialization: Molecular Kinetic Theory of Fluids and Gases\smallskip \\
Department of Hydroaeromechanics \\
Research supervisor: Prof. E.V. Kustova 

{\bf Saint-Petersburg State University} \hfill {\em September 2009 --- June 2013} \\ 
\textbf{Bachelors degree} in Mathematics and Mechanics \smallskip \\
Department of Hydroaeromechanics \\
Research supervisor: Prof. E.V. Kustova
\end{rSection}

{\bf Additional courses completed}
\begin{itemize}
    \item Intensive German language courses (2 blocks of 85 hours), Goethe Institut Berlin, Germany, 2021
    \item ``Computational Research Techniques'' (Applied Parallel Programming, Scientific Visualization, Reproducible Research), Texas Advanced Computing Center, Austin, TX, USA, 2020
    \item ``Machine Learning'', online Stanford University course on Coursera, 2013
\end{itemize}

% \begin{rSection}{Internships} 
%  Runners up in B.G.Shirke Vidyarthi Competition for Innovative Project organized by Pune Construction Engineering Research Foundation in January 2018
% \item Won First Prize in Model Making Competition Organized by Symbiosis Institute of Technology, Pune.
% \end{rSection}
%----------------------------------------------------------------------------------------
%   TECHNICAL STRENGTHS SECTION
%----------------------------------------------------------------------------------------

\begin{rSection}{Technical skills}

\begin{tabular}{ @{} >{\bfseries}l @{\hspace{1ex}} l }
Non-equilibrium flows \ & Numerical modeling, DSMC, DVM, Chapman--Enskog method, \\
\ & uncertainty quantification/sensitivity analysis, CFD \\
Programming languages \ & Python, Fortran, C, C++, Julia, MATLAB, Javascript \\
Tools \ & SPARTA DSMC, DLR TAU, Paraview, Git, Python scientific stack, \\
\ & UQ Toolkit, Trixi.jl \\
Other \ & Machine learning (Scikit-learn, PyTorch) \\ 
% Other \ & Machine learning, digital signal processing \\ 
Languages \ & English (fluent, including technical English) \\
\ & German (B2 level speaking, reading, writing, including technical German) \\
\ & Russian (native speaker)
\end{tabular}

\end{rSection}



\begin{rSection}{Professional Membership}


{\bf COST Network} \hfill {\em March 2022 -- present} \\ 
Member of the ``Mathematical models for interacting dynamics on networks'' (MAT-DYN-NET) networking group, supported by the COST (European Cooperation in Science and Technology) program


{\bf RGD NextGen Group} \hfill {\em Spring 2021 -- present} \\ 
Member of the RGD NextGen Group, aimed at increasing outreach of the Rarefied Gas Dynamics community

{\bf AIAA} \hfill {\em December 2019 -- present} \\ 
Senior AIAA Member \\
Member of the AIAA Thermophysics Technical Committee since spring 2021 \\
Member of Education, Emerging Technologies, Conference subcommittees
% {\bf American Physical Society} \hfill {\em November 2020 -- present}
\end{rSection}
%   EXAMPLE SECTION
%----------------------------------------------------------------------------------------
\newpage

\begin{rSection}{Academic Achievements}

Author and co-author of 22 peer-reviewed publications in SCOPUS/Web of Science-indexed  journals and conference proceedings.
Participant of 18 international and 10 national conferences. 14 invited talks at US, German, and Dutch universities and research institutes.

% Reviewer for
% \begin{enumerate}
%     \item Computers and Fluids
%     \item Acta Astronautica
%     \item Computer Methods in Applied Mechanics and Engineering
%     \item Applied Mathematical Modelling
%     \item Physics of Fluids
%     \item Journal of Thermophysics and Heat Transfer
%     \item MDPI Entropy
%     \item MDPI Fluids
%     \item MDPI Sensors
%     \item AIAA Scitech Conference
%     \item AIAA Aviation Conference
%     \item Rarefied Gas Dynamics Symposium
%     \item STAB/DGLR Symposium Proceedings
% \end{enumerate}
% , , Computer Methods in Applied Mechanics and Engineering, , Physics of Fluids, Journal of Thermophysics and Heat Transfer, AIAA Scitech Conference, AIAA Aviation Conference, Rarefied Gas Dynamics Conference,  MDPI Entropy, MDPI Fluids, MDPI Sensors, STAB/DGLR Symposium Proceedings.

Reviewer for Computers and Fluids, Acta Astronautica, Computer Methods in Applied Mechanics and Engineering, Applied Mathematical Modelling, Physics of Fluids, Journal of Thermophysics and Heat Transfer, AIAA Scitech Conference, AIAA Aviation Conference, Rarefied Gas Dynamics Symposium Proceedings,  MDPI Entropy, MDPI Fluids, MDPI Sensors, STAB/DGLR Symposium Proceedings.

Reviewer of Bachelor and Master theses for Saint-Petersburg State University (2020, 2021) and Peter the Great St. Petersburg Polytechnic University (2021).

{\bf Personal research and travel grants:}


\begin{enumerate}

    \item Alexander von Humboldt Foundation postdoctoral research fellowship, 2021--2023. % (60000 EUR for 24 months, acceptance rate 20%)
    \item Research project ``Improvements of the thermo-chemical relaxation model used by the DLR-TAU code'' jointly sponsored by the DAAD (German Academic Exchange Service) and Saint-Petersburg State University, 2017. % (3300 EUR for 3 months)
    \item Research project ``Influence of variable diameters of vibrationally excited molecules on relaxation processes in strongly non-equilibrium gas flows'' jointly sponsored by the Santander Bank and Saint-Petersburg State University, 2016. % (1400 EUR for 2 weeks)
    \item Research project ``Implementation of models of vibrational transitions in direct simulation methods'', sponsored by the Russian Foundation for Basic Research, 2015. % (4500 EUR for 4 months) 
    \item Saint-Petersburg State University travel grants (2013, 2014, 2017).

\end{enumerate}

{\bf Participation in other research projects}

\begin{enumerate}
    \item 2022--Present: Participant of a Sandia National Laboratories--sponsored project on gap closure simulations.
    \item 2018--2021: Participant of a Sandia National Laboratories--sponsored project on development of new approaches to plasma simulation.
    \item 2015--2016: Participant of the ESA research project ``Exploring angular-momentum phenomenology in aerothermodynamics and MHD''.
    \item 2013--2017: Participant of 3 Saint-Petersburg University grants, 1 Russian Science Foundation grant and 3 Russian Foundation for Basic Research grants.
\end{enumerate}

{\bf Stipends and awards:}
\begin{itemize}
    \item Marie Sk\l{}odowska-Curie Postdoctoral Action Seal of Excellence (2023)
    \item Stipend of the Russian President for students and PhD students studying disciplines corresponding to the prioritized areas of modernization of Russian economy (2017). % (3800 EUR for 12 months)
    \item Stipend of the Russian Government for students and PhD students studying disciplines corresponding to the prioritized areas of modernization of Russian economy (2016). % (1900 EUR for 12 months)
    \item Stipend of the Russian Government for students and PhD students. (2016). % (500 EUR for 12 months)
    \item Stipend of the Russian Government for students and PhD students (2016)
    \item Winner of the Saint-Petersburg Government Grant Competition for Students and Graduate Students (2013). %  (300 EUR)
\end{itemize}
\end{rSection}

\newpage


% \begin{rSection}{Publication list of Dr. Georgii Oblapenko}
{\bf A) Peer-reviewed publications:}
% peer-r. proceedings: 6, 7, 14, 15, 16, 21
\begin{enumerate}

\item Djordji\'c V., Oblapenko G., Pavi\'c-\v Coli\'{c} M., Torrilhon M. (2023): Boltzmann collision operator for polyatomic gases in agreement with experimental data and DSMC method. In: Continuum Mechanics and Thermodynamics, Vol. 35, pp. 103–119.

\item {Oblapenko G.}, Goldstein D., Varghese P., Moore C. (2022): Hedging direct simulation Monte Carlo bets via event splitting. In: Journal of Computational Physics, Vol. 466, p. 111390.

\item Sarna N., Oblapenko G., Torrilhon M. (2021): Moment method for the Boltzmann equation of reactive quaternary gaseous mixture. In: Physica A: Statistical Mechanics and its Applications, Vol. 574, p. 125874.

\item {Oblapenko G.}, Goldstein D., Varghese P., Moore C. (2021): Velocity-space hybridization of DSMC and a quasi-Particle Boltzmann solver. In: Journal of Thermophysics and Heat Transfer, Vol. 35, No. 4, pp. 788-799.

\item {Oblapenko G.}, Kustova E.V. (2020): Influence of angular momentum on transport\\ coefficients in rarefied gases. In: Physica A: Statistical Mechanics and its Applications, Vol. 553, p. 124673.

\item {Oblapenko G.}, Goldstein D., Varghese P., Moore C. (2020): A velocity space hybridization-based Boltzmann equation solver. In: Journal of Computational Physics, Vol. 408, p. 109302.

\item {Campoli L., {Oblapenko G.P.}, Kustova E.V.} (2019): Overview and perspectives of KAPPA library. In: AIP Conference Proceedings, Vol. 2132, No. 1, p. 150005.

\item {{Oblapenko G.P.}, Kustova E.V., Hannemann K., Hannemann. V.} (2019): Assessment of recent thermo-chemical relaxation models using the DLR-TAU code. In: AIP Conference Proceedings, Vol. 2132, No. 1, p. 140006.

\item {{Oblapenko G.P.}} (2018): Calculation of vibrational relaxation times using a kinetic theory approach. In: The Journal of Physical Chemistry A, Vol. 122, No. 50, pp. 9615-9625.

\item {Campoli L., {Oblapenko G.P.}, Kustova E.V.} (2018): KAPPA: Kinetic approach to physical processes in atmospheres library in C++. In: Computer Physics Communications, Vol. 236, pp. 244-267.

\item {Istomin V.A., {Oblapenko G.P.}} (2018): Transport coefficients in high-temperature ionized air flows with electronic excitation. In: Physics of Plasmas, Vol. 25, No. 1, p. 013514.

\item {Kremer G.M., Kunova O.V., Kustova E.V., {Oblapenko G.P.}} (2018): The influence of vibrational state-resolved transport coefficients on the wave propagation in diatomic gases. In: Physica A: Statistical Mechanics and its Applications, Vol. 490, pp. 92-113.

\item {Shoev G., {Oblapenko G.}, Kunova O., Mekhonoshina M., Kustova E.} (2018): Validation of vibration-dissociation coupling models in hypersonic non-equilibrium separated flows. In: Acta Astronautica, Vol. 144, pp. 147-159.

\item {Kustova E.V., Mekhonoshina M.A., {Oblapenko G.P.}} (2017): On the applicability of simplified state-to-state models of transport coefficients. In: Chemical Physics Letters, Vol. 686, pp. 161-166.

\item {{Oblapenko G.P.}, Kashkovsky A.V., Bondar Ye.A.} (2017): State-to-state models of vibrational relaxation in Direct Simulation Monte Carlo (DSMC). In: Journal of Physics: Conference Series, Vol. 815, No. 1, p. 012011.

\item  {Kustova E.V., {Oblapenko G.P.}} (2016): Vibration-dissociation Coupling in Multi-Temperature Viscous Gas Flows. In: AIP Conference Proceedings, Vol. 1786, No. 1, p. 150004.

\item  {Baikov B.S., Bayalina D.K., Kustova E.V., {Oblapenko G.P.}} (2016): Inverse Laplace transform as a tool for calculation of state-specific cross sections of inelastic collisions. In: AIP Conference Proceedings, Vol. 1786, No. 1, p. 090005.

\item  {Shoev G.V., Bondar Ye.A., {Oblapenko G.P.}, and Kustova E.V.} (2016): Development and testing of a numerical simulation method for thermally nonequilibrium dissociating flows in ANSYS Fluent. In: Thermophysics and Aeromechanics, Vol. 23, No. 2, pp. 151-163.

\item {Kustova E.V., {Oblapenko G.P.}} (2016): Mutual effect of vibrational relaxation and chemical reactions in viscous multitemperature flows. In: Physical Review E -- Statistical, Nonlinear, and Soft Matter Physics, Vol 93, No. 3, p. 033127.

\item  {Kustova E.V., Nagnibeda E.A., {Oblapenko G.P.}, Savelev A.S., Sharafutdinov I.Z.} (2016): Advanced models for vibrational–chemical coupling in multi-temperature flows. In: Chemical Physics, Vol. 464, pp. 1-13.

\item {Kustova E.V., {Oblapenko G.P.}} (2015): Reaction and internal energy relaxation rates in viscous thermochemically non-equilibrium gas flows. In: Physics of Fluids, Vol. 27, No. 1, p. 016102.

\item  {Kustova E.V., {Oblapenko G.P.}} (2014): Rates of VT transitions and dissociation and normal mean stress in a non-equilibrium viscous multitemperature N$_2$/N flow. In: AIP Conference Proceedings, Vol. 1628, No. 1, pp. 602-609.

\item  {Kustova E.V., {Oblapenko G.P.}} (2013):
{Normal mean stress and rates of slow process in chemically and vibrationally non-equilibrium multi-temperature gas flows.} In: Vestnik of the Saint-Petersburg University, Vol. 2, p. 111-120. (In Russian).
\end{enumerate}


{\bf B) Publications without peer-review process:}
\begin{enumerate}
    
    \item {Oblapenko G.P.} (2023): Use of Tensor-Train decompositions with a discrete velocity Boltzmann solver. arXiv preprint 2303.15142.
    
    \item {Oblapenko G.P., Hannemann V.} (2023): Sensitivity analysis of non-equilibrium expanding high-enthalpy Flows. In: Proceedings of the AIAA Scitech Forum.

    \item {Oblapenko G., Goldstein D., Varghese P., Moore C.} (2022): Improving PIC-DSMC simulations of RF plasmas via event splitting. In: Proceedings of the European Conference for Aeronautics and Space Sciences (EUCASS), Lille, France.

    \item {Oblapenko G.P., Hannemann V.} (2022): Sensitivity analysis study of expanding hypersonic flows of nitrogen and oxygen. In: Proceedings of the European Conference for Aeronautics and Space Sciences (EUCASS), Lille, France.

    \item {Oblapenko G., Goldstein D., Varghese P., Moore C.} (2022): Improving PIC-DSMC simulations of electrical breakdown via event splitting. In: Proceedings of the AIAA Scitech Forum.
    
    \item {Oblapenko G., Goldstein D., Varghese P., Moore C.} (2021): Modeling of ionized gas flows with a velocity-space hybrid Boltzmann solver. In: Proceedings of the AIAA Scitech Forum.
    
    \item {Oblapenko G., Goldstein D., Varghese P., Moore C.} (2020): Velocity-space hybridization of DSMC and a Boltzmann solver. In: Proceedings of the AIAA Scitech Forum, Orlando, Florida.

    \item {Hannemann K., Schramm J. M., Riedl P., Hannemann V., { Oblapenko G.}} (2018): Thermal non-equilibrium effects on spherically blunted cone flows. In: Proceedings of the 12th International Workshop on Shock Tunnel Technology.
    
    \item Istomin V.A., Kustova E.V., {Oblapenko G.P.} (2017): State-resolved transport properties of electronically excited high-temperature flows behind strong shock waves. In: Proceedings of the 31st International Symposium on Shock Waves, Nagoya, Japan.
    
    \item {{Oblapenko G.P.}, Kashkovsky A.V., Bondar Y.A.} (2017): State-to-state models of physico-chemical processes in direct simulation Monte Carlo (DSMC) computations of 2-dimensional flows. In: Proceedings of the European Conference for Aeronautics and Space Sciences (EUCASS), Milan, Italy.

    \item  {Kustova E.V., {Oblapenko G.P.}, Sharafutdinov I.Z.} (2015):  {Vibrational relaxation models for non-equilibrium multi-temperature flows}. In: Physico-chemical Kinetics in Gas Dynamics, p. 1-10. (In Russian).

    \item  {Kustova E.V.,{Oblapenko G.P.}} (2014): {Vibrational relaxation rates in multi-temperature gas flows.} In: Physico-chemical Kinetics in Gas Dynamics, p. 1-4. (In Russian).
\end{enumerate}


{\bf Invited seminar talks/lectures:}
\begin{enumerate}
    \item Seminar, Institut f\"{u}r Raumfahrtsysteme, Universit\"{a}t Stuttgart, 22 November 2022 (Stuttgart, Germany)
    \item Department Seminar, Institute of Aerodynamics and Flow Technology, German Aerospace Center (DLR), 2 November 2022 (G\"{o}ttingen, Germany)
    \item Department Seminar, Institute of Aerodynamics and Flow Technology, German Aerospace Center (DLR), 7 September 2022 (G\"{o}ttingen, Germany)
    \item Bernoulli Institute for Mathematics, Computer Science and Artificial Intelligence, Rijksuniversiteit Groningen, 8 August 2022 (Groningen, Netherlands)
    \item Steinbuch Center for Computing, Karlsruhe Institute of Technology (KIT), 16 March 2022 (Karlsruhe, Germany)
    % \item Department of Aeronautics and Astronautics, Massachusets Institute of Technology (MIT), 03 March 2022 (Cambridge, MA, USA)
    \item Department Seminar, Institute of Aerodynamics and Flow Technology, German Aerospace Center (DLR), 23 February 2022 (G\"{o}ttingen, Germany)
    \item Center for Computational Engineering Science, Mathematics Division, RWTH Aachen, 20 January 2022 (Aachen, Germany)
    \item RGD NextGen Online Seminar Series, 4 June 2021
    \item Department of Aerospace Engineering and Engineering Mechanics, University of Texas at Austin, 15 April 2021 (Austin, TX, USA)
    \item PECOS Center, Oden Institute for Computational Engineering and Sciences, University of Texas at Austin, 8 March 2021 (Austin, TX, USA)
    \item Center for Computational Engineering Science, Mathematics Division, RWTH Aachen, 12 November 2019 (Aachen, Germany)
    \item Department of Aerospace Engineering and Engineering Mechanics, University of Texas at Austin, 28 February 2019 (Austin, TX, USA)
    \item Oden Institute for Computational Engineering and Sciences, University of Texas at Austin, 22 February 2019 (Austin, TX, USA)
    \item Department Seminar, Institute of Aerodynamics and Flow Technology, German Aerospace Center (DLR), 8 January 2018 (G\"{o}ttingen, Germany)
\end{enumerate}
% \end{rSection}


% Conferences (intl): 

% 1.  MMKT 2023, Karlsruhe
% 2.  SIAM CSE23
% 3.  AIAA Scitech 2023
% 4.  EUCASS 2022
% 5.  Humboldt Networking meeting, 8-10 November 2021.
% 6.  Pre-RGD workshop, 7-10 July 2021
% 7.  AIAA Scitech, 2021
% 8.  APS DFD Meeting, 2020
% 9.  AIAA Scitech, 2020 (Orlando, FL, USA)
% 10. DSMC Workshop, 2019 (Santa Fe, NM, USA)
% 11. Workshop on Modern Inverse Problems, 2019 (Austin, TX, USA)
% 12. Tallinn University course on Experimental Interaction Design (Physiological Computing Technologies for Performative Arts), 2018 (Saint-Petersburg, Russia)
% 13. International conference ``7th European Conference for Aeronautics and Space Sciences'', 2017 (Milan, Italy)
% 14. International EUCASS workshop ``Aerospace Thematic Workshops: Fundamentals of Aerodynamic Flow and Combustion Control by Plasmas'', 2017 (Pushkin, Russia)
% 15. International conference ``30th International Symposium on Rarefied Gas Dynamics'', 2016 (Victoria BC, Canada)
% 16. International conference ``International Scientific Conference On Mechanics ``The Seventh Polyakhonv's Reading'''', 2015 (Saint-Petersburg, Russia)
% 17. International conference ``29th International Symposium on Rarefied Gas Dynamics'', 2014 (Xi'an, China)
% 18. International conference ``9th IFAC Symposium on Advances in Control Education'', 2012 (Nizhniy Novgorod, Russia)

% Conferences (German):
% 1.  STAB Symposium, 2022 (Berlin, German)
% 2.  STAB Workshop, 2021 (Göttingen, Germany)
% 3.  STAB Workshop, 2019 (Göttingen, Germany)


% Conferences (Russian):
% 1.  All-Russian conference on hydroaeromechanics, dedicated to S.V. Vallander's 100th anniversary, 2017 (Saint-Petersburg, Russia)
% 2.  All-Russian school-seminar ``Aerophysics and physical mechanics of classical and quantum systems'', 2016 (Moscow, Russia)
% 3.  All-Russian school-seminar ``Aerophysics and physical mechanics of classical and quantum systems'', 2015 (Moscow, Russia)
% 4.  All-Russian seminar ``XXIV All-Russian seminar with international partnership on jet, separation, and non-stationary flows'', 2015 (Novosibirsk, Russia)
% 5.  All-Russian school-seminar ``Aerophysics and physical mechanics of classical and quantum systems'', 2014 (Moscow, Russia)
% 6.  All-Russian conference ``Modern problems in rarefied gas dynamics'', 2013 (Novosibirsk, Russia)

\end{document}
